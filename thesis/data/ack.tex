\begin{ack}
  在读研前,一位老师向我形容读研的感受:“本科的时候,我不知道知识的门在哪里。读
  完研,我看到了那扇门。”怀着对知识的向往与崇敬,我也踏上了寻门之路。三年的读研
  时光,给了我一个静心学习的机会。临近毕业之际,重新品读这句话,幡然醒悟:知识的
  大门,就在灯火阑珊处。只有戒骄戒躁,踏实研究,才能发现这扇神奇的门。

  感谢把我带到知识大门的人,尤其是我的恩师李兴民教授。李教授为人谦逊热忱,治学严
  谨,不但有深厚的学术造诣,对学生的关怀也无微不至。在他的悉心指导下,我不仅在学
  术研究上有所长进,更在为人处世方面受益匪浅。在毕业论文写作期间,李教授给我提出
  了很多宝贵意见,帮助我顺利完成论文工作。今后,我会谨记李教授对我的指导,学习他
  乐观豁达的心境,铭记“知足知不足,有为有不为”的教诲,不负师恩。

  感谢计算机学院一路走来指导过我的其他老师,尤其是鲍苏苏教授,单志龙教授、王立斌
  副教授、陈寅副教授以及张奇支教授,他们是我求知路上的明灯,用渊博的学识授人以渔。
  感谢我的师母王敬老师以及辅导员谢子娟老师,她们在生活和工作中给予我细心爱护和帮
  助,
  
  感谢深圳先进技术研究院的老师和同学,
  
  求知的道路是孤独的。很感谢
  
  感谢师母王敬老师,
  
  李老师,师母,鲍老师;
  王立斌,陈寅,单志龙,张奇支;
  谢子娟,院领导;
  同门,舍友,赵世豪,徐泽坤,雷楚楚;
  父母家人;
  陈宝权,Daniel Cohen-Or,Oliver Deaussen,Andrei Sharf,Daniel Ron;
  开源,Github,OpenCV。

  “吾辈读书只有两事,一者进德之事,以图无忝所生;一者修业之事,以图自卫其身。”
 
  学海无涯,三年的研究生时光,让我看到了知识的大门,但这只是一个起点。我的导师李
  兴民教授曾在他的课上写了一行“路漫漫其修远兮”的诗句送给我们。活到老,学到老。

%%% Local Variables:
%%% mode: latex
%%% TeX-master: "../thesis"
%%% End:

