\documentclass{article}

\usepackage{tikz}
\usepackage[pdftex,active,tightpage]{preview}

\setlength\PreviewBorder{0mm}
\PreviewEnvironment{tikzpicture}

\usetikzlibrary{arrows,shapes.misc,positioning,calc,chains,scopes}

\tikzstyle{neuron}=[circle,fill=black!25,minimum size=12,inner sep=4]
\tikzstyle{input neuron}=[neuron, fill=green!25, draw=green!70]
\tikzstyle{output neuron}=[neuron, fill=blue!25, draw=blue!70]
\tikzstyle{hidden neuron}=[neuron, fill=red!25, draw=red!70]
\tikzstyle{small neuron}=[circle, minimum size=6, inner sep=2, fill=blue!25, draw=blue!70]
\tikzstyle{A neuron}=[circle, minimum size=6, inner sep=0, fill=green!25, draw=green!70]
\tikzstyle{B neuron}=[circle, minimum size=6, inner sep=0, fill=red!25, draw=red!70]
\tikzstyle{ouput box}=[rectangle, fill=blue!25, draw=blue!70, minimum size=12, inner sep=4]
\tikzstyle{hidden box}=[rectangle, fill=red!25, draw=red!70, minimum size=12, inner sep=4]
\tikzstyle{thickarrow}=[thick,>=stealth]
\tikzstyle{connect}=[thin]

\begin{document}

\begin{tikzpicture}[node distance=1.6cm and 1cm]

  \node (output1) [output neuron, label=left:$\omega_1$] {};
  \node (outputj) [output neuron, label=left:$\omega_j$, right=of output1] {};
  \node (outputm) [output neuron, label=left:$\omega_m$, right=of outputj] {};

  \node (inputx1) [input neuron, below=of output1, xshift=-0.8cm]{};
  \node (inputx2) [input neuron, right=of inputx1]{};
  \node (inputxi) [input neuron, right=of inputx2]{};
  \node (inputxn) [input neuron, right=of inputxi]{};

  \node (label1)[above=of output1, yshift=-1.0cm]{$o_1$};
  \node (labelj)[above=of outputj, yshift=-1.0cm]{$o_j$};
  \node (labelm)[above=of outputm, yshift=-1.0cm]{$o_m$};

  \node [right=of label1, xshift=-0.8cm]{\ldots};
  \node [right=of labelj, xshift=-0.8cm]{\ldots};

  \node (labelx1) [below=of inputx1, yshift=1.0cm]{$x_1$};
  \node (labelx2) [below=of inputx2, yshift=1.0cm]{$x_2$};
  \node (labelxi) [below=of inputxi, yshift=1.0cm]{$x_i$};
  \node (labelxn) [below=of inputxn, yshift=1.0cm]{$x_n$};

  \node [right=of labelx2, xshift=-0.8cm]{\dots};
  \node [right=of labelxi, xshift=-0.8cm]{\ldots};

  \draw [thickarrow,->] (output1) -- (label1);
  \draw [thickarrow,->] (outputj) -- (labelj);
  \draw [thickarrow,->] (outputm) -- (labelm);

  \draw [thickarrow,<-] (inputx1) -- (labelx1);
  \draw [thickarrow,<-] (inputx2) -- (labelx2);
  \draw [thickarrow,<-] (inputxi) -- (labelxi);
  \draw [thickarrow,<-] (inputxn) -- (labelxn);

  \draw [thin] (output1)--(inputx1);
  \draw [thin] (output1)--(inputx2);
  \draw [thin] (output1)--(inputxi);
  \draw [thin] (output1)--(inputxn);

  \draw [thin] (outputj)--(inputx1);
  \draw [thin] (outputj)--(inputx2);
  \draw [thin] (outputj)--(inputxi);
  \draw [thin] (outputj)--(inputxn);

  \draw [thin] (outputm)--(inputx1);
  \draw [thin] (outputm)--(inputx2);
  \draw [thin] (outputm)--(inputxi);
  \draw [thin] (outputm)--(inputxn);
  
\end{tikzpicture}

\end{document}

%%% Local Variables: 
%%% mode: latex
%%% TeX-master: t
%%% End: 
